\documentclass{article}

\usepackage{amsthm}
\usepackage{amsmath}
\usepackage{amssymb}
\usepackage{amsfonts}

\usepackage{natbib}
\usepackage{hyperref}
\bibliographystyle{chicago}

\usepackage{graphicx}

\title{Notes for thesis}
\date{\today}
\author{Sang Woo Park}

\begin{document}
\maketitle

\section{Model assumptions}

The TSIR model:
\begin{equation}
\begin{aligned}
S_{t+1} &= B_t + S_t - \beta S_t \frac{I_t^\alpha}{N_t}\\
I_{t+1} &= \beta S_t \frac{I_t^\alpha}{N_t}\\
\end{aligned}
\end{equation}
where $\alpha$ is a phenomenological parameter that accounts for heterogeneity and discretization of a system.

An implicit term in this model is $\Delta T$, time difference between step $t$ and step $t+1$. The underlying assumption is that disease generation is equal to reporting period. To consider wide range of models, we model $\Delta t$ explicitly:
\begin{equation}
\begin{aligned}
S_{t+\Delta t} &= B_t + S_t - \beta S_t \frac{I_t^\alpha}{N_t} \Delta t\\
I_{t+\Delta t} &= \beta S_t \frac{I_t^\alpha}{N_t} \Delta t\\
\end{aligned}
\end{equation}

\subsection{Force of infection}

Force of infection is defined as the rate at which a susceptible individual acquire infection.
In discrete time models, it is more natural to think of probability of infection in time $\Delta t$.
Given force of infection, $\phi(t)$, the probability that an individual that is susceptible at time $t_0$ will be infected before $t_0 + \Delta t$ is given by
\begin{equation}
1 - \exp\left(-\int_{t_0}^{t_0 + \Delta t} \phi(s) ds\right).
\end{equation}
Typically, it is assumed that the force of infection stays constant over $\Delta t$ and probability of infection is modeled as $1 - \exp(- \phi_{t_0} \Delta t)$ where $\phi_t = \beta I_t^\alpha/N_t$ for the SIR model.
TSIR model, on the other hand, assumes that probability of infection is given by $\beta I_t^\alpha/N_t$.

For measles outbreak in London, we find that both representations give very similar estimates of probability of infection (correlation coefficient between the estimates is greater than 0.99).

\subsection{Infectious period and generation interval}

Probability that an infected individual recovers in $\Delta t$ time unit is given by
\begin{equation}
1 - \exp (- \lambda \Delta t).
\end{equation}
Then, the dynamics of infected individuals can be written as
\begin{equation}
\begin{aligned}
S_{t+\Delta t} &= B_t + S_t - \beta S_t \frac{I_t^\alpha}{N_t} \Delta t\\
I_{t+\Delta t} &= \beta S_t \frac{I_t^\alpha}{N_t} \Delta t + I_t - I_t \left(1 - \exp (- \lambda \Delta t) \right) \\
\end{aligned}
\end{equation}
In this model, the mean infectious period as well as the mean generation interval is equal to $1/(1 - \exp (- \lambda \Delta t))$.


\end{document}
