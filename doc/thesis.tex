\documentclass{article}

\usepackage{amsthm}
\usepackage{amsmath}
\usepackage{amssymb}
\usepackage{amsfonts}

\usepackage{natbib}
\usepackage{hyperref}
\bibliographystyle{chicago}

\usepackage{graphicx}

\title{Thesis}
\date{\today}
\author{Sang Woo Park}

\begin{document}
\maketitle


\section{Introduction}

Please write something here.

\section{Deterministic SIR model}

First, we begin our exploration using the simplest epidemic model.
The susceptible-infected-recovered (SIR) model describes...:
\begin{equation}
\begin{aligned}
\frac{dS}{dt} &= - \beta S \frac{I}{N}\\
\frac{dI}{dt} &= \beta S \frac{I}{N} - \gamma I\\
\frac{dR}{dt} &= \gamma I
\end{aligned}
\label{eq:sir}
\end{equation}

We want to consider discretized version of this model.

\subsection{Probability of infection}


\subsection{Infectious period}

Geometric distribution with probability of $1-\exp(-\gamma \Delta t)$ and unit of $\Delta t$.
Then, mean infectious period and generation interval is given by 
\begin{equation}
\frac{\Delta t}{1-\exp(-\gamma \Delta t)}.
\end{equation}
An important but often overlooked component is the variance of the distribution.
Squared coefficient of variation for this distribution is equal to $\exp(-\gamma \Delta t)$, which necessarily depends on pre-specified time-step $\Delta t$.
If we want to match the mean of the distribution to a fixed value $\mu$ regardless of our choices of $\Delta t$, we obtain 
\begin{equation}
\hat \gamma = - \frac{1}{\Delta t} \log\left(1 - \frac{\Delta t}{\mu}\right)
\end{equation}
Then,
\begin{equation}
\mathrm{CV}_{\tiny\textrm{Infectious period}}(\Delta t)^2 = 1 - \frac{\Delta t}{\mu}.
\end{equation}

We expect changes in variation in infectious period to affect variation in stochstic realizations.

This means that the relationship between $r$ and $\mathcal R$ changes.
For this generation-interval distribution, we have
\begin{equation}
\mathcal R = 1 + \frac{\exp(r) -1}{(1-\exp(-\gamma \Delta t)) }
\end{equation}
Substituting 
\begin{equation}
r = \frac{1}{\Delta t} \log \left(1 + \beta - \frac{\Delta t}{\mu}\right),
\end{equation}
as well as $\hat \gamma$, we get
\begin{equation}
\mathcal R = 1 + \frac{\mu}{\Delta t} \left(1 + \beta - \frac{\Delta t}{\mu}\right)^{1/\Delta t}
\end{equation}

We find that the formula for reproduction number suggested by \cite{he2009plug} only holds when generation interval is held constant.


First, we compare how the dynamics of a system depend on the value of $\Delta t$?

\subsection{Incidence, prevalence, and mortality}






\section{Bibliography}

\bibliography{thesis}

\section{Appendix}

\subsection{Derivation of growth rate in discrete-time SIR model}


Derivation:
\begin{equation}
\begin{aligned}
S_{t+\Delta t} &= S_t - S_t (1- \exp(-\beta I_t \Delta t/N))\\
I_{t+\Delta t} &= S_t (1- \exp(-\beta I_t \Delta t/N)) + I_t - I_t (1- \exp(-\gamma \Delta t))
\end{aligned}
\end{equation}
Assuming tat $S_t$ is approximately equal to $N$, we get
\begin{equation}
I_{t+\Delta t} = N_t (1- \exp(-\beta I_t \Delta t/N)) + I_t \exp(-\gamma \Delta t)
\end{equation}
When $I_t \approx 0$, $\exp(-\beta I_t \Delta t/N) \approx 1 - \beta I_t \Delta t/N$ and so
\begin{equation}
\begin{aligned}
I_{t+\Delta t} &\approx  I_t \beta \Delta t + I_t \exp(-\gamma \Delta t)\\
&= I_t (\beta \Delta t + \exp(-\gamma \Delta t))
\end{aligned}
\end{equation}
Substituting $\hat{\gamma}$, we get
\begin{equation}
I_{t+\Delta t} = I_t \left(1 + \beta - \frac{\Delta t}{\mu}\right),
\end{equation}
Therefore,
\begin{equation}
I_{t} = I_0 \left(1 + \beta - \frac{\Delta t}{\mu}\right)^{t/\Delta t}
\end{equation}
and the initial growth rate is given by
\begin{equation}
r = \frac{1}{\Delta t} \log \left(1 + \beta - \frac{\Delta t}{\mu}\right).
\end{equation}




\end{document}
