\documentclass{article}

\usepackage{amsthm}
\usepackage{amsmath}
\usepackage{amssymb}
\usepackage{amsfonts}

\usepackage{natbib}
\usepackage{hyperref}
\bibliographystyle{chicago}

\usepackage{graphicx}

\title{Thesis}
\date{\today}
\author{Sang Woo Park}

\begin{document}
\maketitle


\section{Introduction}

Fitting discrete-time models.

\section{SIR model}

The susceptible-infected-recovered (SIR) model is one of the simplest epidemic models which describe how disease spreads in a homogeneously mixing population.
In general, the SIR model can be written as 
\begin{equation}
\begin{aligned}
\frac{dS}{dt} &= b(t) - \phi(t) S\\
\frac{dI}{dt} &= \phi(t) S - \gamma I\\
\frac{dR}{dt} &= \gamma I
\end{aligned}
\label{eq:sir}
\end{equation}
where $S$, $I$, and $R$ describe the number of susceptible, infected, and recovered individuals.
$b(t)$ represents recruitment rate to the susceptible population, and $\gamma$ represents per-capita recovery rate.
$\phi(t)$ represents force of infection, which is defined as the rate at which a susceptible individual acquires infection.
Typically, we write $\phi(t) = \beta(t) I/N$ to represent \textit{frequency}-dependent action or $\phi(t) = \beta I$ to represent \textit{density}-dependent action, where $N = S + I + R$ is the total population size.
%% heterogeneity and what not

\begin{itemize}
	\item Underlying system is stochastic
	\item Data is discrete
\end{itemize}


\subsection{Incidence, prevalence, and mortality}

Data often comes in discrete counts and we want to distinguish among different kinds of data. In particular, we want to distinguish between incidence, prevalence, and mortality.
Incidence is defined as the number of newly infected cases.
\begin{equation}
i(t, t+\Delta t) = \int_{t}^{t+\Delta t} \phi(s) S ds
\end{equation}
Prevalence is defined as the number of infected individuals present in the population.
Finally, mortality is defined as the number of individuals that died over a time period
\begin{equation}
m(t, t+\Delta t) = \int_{t}^{t+\Delta t} \gamma I ds
\end{equation}



\section{Discrete-time SIR model}

\subsection{Time-series SIR model}

TSIR:
\begin{equation}
\begin{aligned}
S_{t + \Delta t} &= B_t + S_t - \beta_t S_t \frac{I_t}{N_t} \Delta t\\
I_{t + \Delta t} &= \beta_t S_t \frac{I_t}{N_t} \Delta t
\end{aligned}
\end{equation}
Assumptions are:
\begin{itemize}
	\item Fixed generation time
	\item Linear probability of infection
\end{itemize}

\subsection{Hazard-based model}

Hazard-based model:
\begin{equation}
\begin{aligned}
S_{t + \Delta t} &= B_t + S_t - S_t \left(1- \exp\left(-\beta_t \frac{I_t}{N_t} \Delta t\right)\right)\\
I_{t + \Delta t} &= S_t \left( 1- \exp\left(-\beta_t \frac{I_t}{N_t} \Delta t\right)\right) + I_t - I_t (1 - \exp(-\gamma \Delta t))
\end{aligned}
\end{equation}
Tries to account for variation in ... but still ... When are these assumptions appropriate?

\subsection{Generation-based model}





\subsection{Probability of infection}

In both the TSIR model and the hazard-based model, transition from the susceptible comparment, $S$, to the infected compartment, $I$, is represented as a product of number of susceptible individuals and probability of infection between two time steps $t$ and $t + \Delta t$.
The TSIR model assumes that the probability of infection is a linear function of prevalence $I_t$.
This formulation is based on the Euler approximation to the solution of the ordinary differential equation.
On the other hand, the hazard-based model assumes that the probability of infection is an inverse exponential function of prevalence $I_t$.
Besides their differences in the relationship between prevalence and probability of infection, there are two more assumptions that we need to consider: (1) force of infection remains constant over two time steps and (2) new susceptible individuals have zero probability of infection between the two time steps.



\subsection{Infectious period}

Geometric distribution with probability of $1-\exp(-\gamma \Delta t)$ and unit of $\Delta t$.
Then, mean infectious period and generation interval is given by 
\begin{equation}
\frac{\Delta t}{1-\exp(-\gamma \Delta t)}.
\end{equation}
An important but often overlooked component is the variance of the distribution.
Squared coefficient of variation for this distribution is equal to $\exp(-\gamma \Delta t)$, which necessarily depends on pre-specified time-step $\Delta t$.
If we want to match the mean of the distribution to a fixed value $\mu$ regardless of our choices of $\Delta t$, we obtain 
\begin{equation}
\hat \gamma = - \frac{1}{\Delta t} \log\left(1 - \frac{\Delta t}{\mu}\right)
\end{equation}
Then,
\begin{equation}
\mathrm{CV}_{\tiny\textrm{Infectious period}}(\Delta t)^2 = 1 - \frac{\Delta t}{\mu}.
\end{equation}

We expect changes in variation in infectious period to affect variation in stochstic realizations.

This means that the relationship between $r$ and $\mathcal R$ changes.
For this generation-interval distribution, we have
\begin{equation}
\mathcal R = 1 + \frac{\exp(r) -1}{(1-\exp(-\gamma \Delta t)) }
\end{equation}
Substituting 
\begin{equation}
r = \frac{1}{\Delta t} \log \left(1 + \beta \Delta t - \frac{\Delta t}{\mu}\right),
\end{equation}
as well as $\hat \gamma$, we get
\begin{equation}
\mathcal R = 1 + \frac{\mu}{\Delta t} \left(1 + \beta \Delta t - \frac{\Delta t}{\mu}\right)^{1/\Delta t}
\end{equation}
Then, we can obtain a relationship between $\mathcal R$ of a discrete-time system and that of a corresponding continuous time system, assuming that contact rate $\beta$ and mean generation-time $\mu$ are the same:
\begin{equation}
\mathcal R_{\tiny \textrm{discrete}} = 1 + \frac{\mu}{\Delta t} \left(1 + \frac{(\mathcal R_{\tiny \textrm{continuous}}  - 1) \Delta t }{\mu}\right)^{1/\Delta t}
\end{equation}

On the other hand, we may want to match choose $\beta$ and $\mu$ to match true $r$ and $\mathcal R$.



% We find that the formula for reproduction number suggested by \cite{he2009plug} only holds when generation interval is held constant.

First, we compare how the dynamics of a system depend on the value of $\Delta t$?



\section{Model fitting}

\begin{itemize}
	\item Trajectory matching
	\item Gradient matching
	\item Generalized profiling
	\item TSIR
	\item Particle filter
	\item Synthetic likelihood
\end{itemize}

\subsection{Trajectory matching}

\subsection{Gradient matching}

When incidence is reported, we have
\begin{equation}
E[y_t] = \rho \int_{t}^{t+\Delta t} \phi(s) S ds \approx \rho \phi(s) S \Delta t
\end{equation}
So in some sense, we already have information about the infection rate. 
We have indirect observation of the states.

We can use a semi-parametric approach.


\subsection{Generalized profiling}

\begin{equation}
\sum \mathrm{distrib}(y_t, \mu_t, ...) + \lambda \sum 
\end{equation}




\section{Bibliography}

\bibliography{thesis}

\section{Appendix}

\subsection{Derivation of growth rate in discrete-time SIR model}


Derivation:
\begin{equation}
\begin{aligned}
S_{t+\Delta t} &= S_t - S_t (1- \exp(-\beta I_t \Delta t/N))\\
I_{t+\Delta t} &= S_t (1- \exp(-\beta I_t \Delta t/N)) + I_t - I_t (1- \exp(-\gamma \Delta t))
\end{aligned}
\end{equation}
Assuming tat $S_t$ is approximately equal to $N$, we get
\begin{equation}
I_{t+\Delta t} = N_t (1- \exp(-\beta I_t \Delta t/N)) + I_t \exp(-\gamma \Delta t)
\end{equation}
When $I_t \approx 0$, $\exp(-\beta I_t \Delta t/N) \approx 1 - \beta I_t \Delta t/N$ and so
\begin{equation}
\begin{aligned}
I_{t+\Delta t} &\approx  I_t \beta \Delta t + I_t \exp(-\gamma \Delta t)\\
&= I_t (\beta \Delta t + \exp(-\gamma \Delta t))
\end{aligned}
\end{equation}
Substituting $\hat{\gamma}$, we get
\begin{equation}
I_{t+\Delta t} = I_t \left(1 + \beta \Delta t - \frac{\Delta t}{\mu}\right),
\end{equation}
Therefore,
\begin{equation}
I_{t} = I_0 \left(1 + \beta\Delta t - \frac{\Delta t}{\mu}\right)^{t/\Delta t}
\end{equation}
and the initial growth rate is given by
\begin{equation}
r = \frac{1}{\Delta t} \log \left(1 + \beta \Delta t - \frac{\Delta t}{\mu}\right).
\end{equation}






\end{document}
